\documentclass[a4paper,11pt]{article}
\usepackage{url}
\usepackage{hyperref}

\input maths

\hoffset 0in
\oddsidemargin 0in
\voffset -0.4in
\topmargin 0in
\headheight 12pt
\headsep 0,5in

\marginparwidth 50pt
\marginparsep 5pt
\reversemarginpar


\textwidth 6.5in
\displaywidth 6.5in
\textheight 240mm
\parindent 0mm
\parskip \baselineskip

\newcommand{\code}[1]{\texttt{#1}}

%\newcommand{\heading}[1]{\vspace{2ex}\section*{#1}}
\newcommand{\refsection}[1]{Section \ref{#1}}


\begin{document}

\title{ \bf ENEL464 Embedded Software and Advanced Computing 2019 \\ Group Assignment 2}
\author{Michael Hayes}
\date{}
\maketitle


\section{Introduction}

The purpose of this assignment is to implement a numerical algorithm
on a desktop computer to make the most efficient use of its caches and
multiple cores.

The algorithm is Jacobi relaxation.  This is an iterative algorithm
used to approximate differential equations, for example, Poisson's and
Laplace's equations.  Poisson's equation can be used to find the
electric potential given a specified charge distribution or
temperature given a specified heat source.  There are more efficient
ways to solve this problem using Green's functions and Fourier
transforms but that is not the purpose of this assignment.

\section{Jacobi relaxation}

The discrete form of Poisson's equation,
%
\begin{equation}
  \nabla^2 V_{i,j,k,n} = f_{i,j,k,n},
\end{equation}
%
can be solved iteratively using Jacobi relaxation, where
%
\begin{equation}
  V_{i,j,k,n+1} = \frac{1}{6}\encp{V_{i+1,j,k,n} + V_{i-1,j,k,n} + V_{i,j+1,k,n} + V_{i,j-1,k,n} + V_{i,j,k+1,n} + V_{i,j,k-1,n} - \Delta^2 f_{i,j,k}}.
\label{eqn:Jacobi}
\end{equation}
%
Here $\Delta = \Delta x = \Delta y = \Delta z$ is the spacing between
voxels in metres, $0 \le i < N$, $0 \le j < N$, and $0 \le k < N$.
Voxels on the boundary ($i = -1$, $i=N$, $j = -1$, $j=N$, $k = -1$,
$k=N$) have a value $V_{\mathrm{bound}}$.  This is a Dirichlet
boundary condition, equivalent to enclosing the problem in a metal box
at potential $V_{\mathrm{bound}}$.

\section{Implementation}

Implement an algorithm for Jacobi relaxation in either C or C++.  Your
goal is to find a fast implementation that will run on a CAE lab
computer, making best use of the caches and multiple cores.

Your implementation needs to work with an arbitrary 3-D source
distribution $f$.  The potential $V$ needs to be stored as a double
data type.



\section{Testing}

Test your algorithm with a single point charge in the centre of the
volume, i.e.,
%
\begin{equation}
  f_{i,j,k} = \left\{
  \begin{array}{ll}
    1 & i=N/2, j=N/2, k=N/2, \\
    0 & \mbox{otherwise}
  \end{array}\right.,
\end{equation}
%
where the volume is comprised of $N \times N \times N$ voxels.  Note,
you algorithm must work with arbitrary source distributions.



%% There is a simple explanation of the 2-D algorithm at
%% \url{https://blogs.msdn.microsoft.com/visualizeparallel/2010/03/29/the-jacobi-relaxation-an-instance-of-data-parallelism/}

\section{Support}

Only questions submitted via the ENCE464 assignment forum will be
answered.  Emails will be quietly ignored.


\section{Reports}

The reports are to be submitted as PDF documents through the ENCE464
Learn page.  They will be submitted to TurnItIn for plagiarism
checking.

Guidelines for writing a report are available at\\
\url{https://eng-git.canterbury.ac.nz/mph/report-guidelines/blob/master/report-guidelines.pdf}.

Each report is to use a 12 point font and be no longer than four
pages, including appendices.

%Ensure in your report that you discuss your implementation in terms of
%cache usage and core usage.  You should present profiling information
%showing which parts of your program takes the most time to execute.

Your report will be marked in terms of:
%
\begin{itemize}
\item Written style
\item Presentation
\item Cache analysis
\item Multithreading analysis
\item Profiling analysis 
\item Optimisation analysis
\item Overall excellence  
\end{itemize}

Your report should present the average time and standard deviation of
the time to run 500 iterations of your implementation for $N=101, 201,
301, 401, 501, 601, 701, 801, 901$.

%% Bonus marks will be awarded for implementations that handle Neumann
%% boundary conditions on each boundary.  This is equivalent to an
%% insulating layer on the boundary.


\section{Code}

You must submit your code as a \code{.zip} file so it can be checked
for numerical accuracy and plagiarism.  Your fastest implementation
must be able to built by running \code{make}.


\end{document}
